\section{Études de l'existant}
Commentaire.  Cette partie ne doit être complétée que si il existe déjà une application et que celles-ci doit être adaptée ou modifiée.  C’est la référence à partir de laquelle nous baserons nos travaux pour atteindre la situation demandée par le client.  On place dans cette section toutes les informations utiles pour la compréhension de la situation de départ.  Par exemple quand on fait la migration d’une base de données, quand on ajoute de nouvelles fonctionnalités.  Si on commence une toute nouvelle application, il ne faut pas tenir compte d’une situation existante.  Dans ce cas, indiquez le dans votre analyse afin de faire savoir que vous n’avez pas oublié.

\subsection{Contexte de l'analyse}
Commentaire.  Décrivez en quelques mots le contexte dans lequel le nouveau programme, service … devra être intégré.  Il s’agit en général de fonctionnalités existantes, de références aux activités économiques du client ou du demandeur.

\subsection{Modèle de données de l'existant}
Commentaire.  Décrivez ici le modèle de donnée qui existe déjà, indiquez les parties qui seront modifiées.

\subsection{Modèle de traitement de l'existant}
Commentaire.  Décrivez les programmes, les traitements de la situation actuelle, vous pouvez utiliser un diagramme UML ou un diagramme de flux de données (DFD).